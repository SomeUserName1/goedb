\begin{abstract}
\begin{center}
    \textbf{Abstract:}
\end{center}{}
Graphs are omnipresent in our world. 
Not only geographic maps and social networks, but also biological systems, like brains and the spreading of diseases are modelled using graphs. \\
Graph traversals are often used to examine such graphs, e.g. to find shortest paths between two places or to compute differential equation based spreading processes. \\
Databases provide means to store large amounts of data reliable and scalable. 
As the bottleneck to data-intensive information processing in current computing systems is the amount of time spent to read and write data to secondary storage, this is commonly one of the most crucial aspects of scalability. \\
While relational databases have been optimized for decades, graph databases are a relatively new branch of research in this respect.
In order to optimize the performance of traversal based queries, the number of disk accesses needs to be minimized. 
This is achieved by leveraging what is called locality of reference.
One way to achieve this is to rearrange records such that, when these are accessed together, they are also stored together. \\
A survey of state of the art graph record rearrangement strategies is presented, along with the proposition of a new improved method.
Finally, the methods are implemented and evaluated against a pragmatic metric to measure the scalability of traversal-based algorithms.
\end{abstract}

\newpage
\section*{Acknowledgements}
I owe an enormous debt to Michael Grossniklaus. 
As a mentor, he always provided me with guidance, tolerance, and support generously whenever I was struggling.

I cannot overexpress my gratitude towards my parents.
They raised me to the person that I am.
Only their support made it possible to study my passion.

Further, I want to thank my siblings Leo Klopfer, Jasmin Wetzel, her husband Marius and my girlfriend Natascha Reddemann for always being there for me and having an ear open when the times were stormy.

Working with colleagues and spending times with friends greatly augmented my time here in Konstanz. Thanks to Stephan Perren, Dario Graf, Jannik Bamberger, Leo Wörteler, Manuel Hotz and many others.


Finally, I'd like to thank Theodoros Chondrogiannis for the inspiration, innumerable discussions, his clearness, and the ability to keep me focused. 
You taught me how to approach things scientifically --- not intending to make you responsible for all non-sense I produce, of course.
\newpage
