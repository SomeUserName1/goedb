\chapter{Introduction}\label{\positionnumber}
Graph-like strucutres are omnipresent in out world:
Places, like cities and roads like highways are naturally modelled using graphs.
When thinking about social structures, people can be modelled as nodes and interactions as edges. 
In dynamical systems like the brain, the structure of the system can be modelled as a graph and the dynamics of the process can be modelled as algorithms on the graph structure like the spreading activation algorithm and by changes to the network itself~\autocite{anderson, dayan1991reinforcing}.
The internet and routing protocols used by the hardware nodes rely on graph theory to optimize the flow of information~\autocite{bgp}.
Especially in the recent years, modelling pandemic dynamics using graph-based models, like the susceptible-infected-recovered and the susceptible-infected-susceptible model, has gained a lot of attention~\autocite{kermack1927contribution, dawood2012estimated, sridhar2020modelling, chang2020modelling}.

As the graphs grow larger and more complex, the need to store it reliable, maintainable and scalable emerges.
In addition to accountability issues, databases provide exceptional performance for some operations. 
In the context of graphs, one set of operations that are executed frequently in broad range of problems are traversal-based queries.
Given a knowledge graph, if we want to retrieve related concepts to a given one, then a breadth first traversal is appropriate. If you want to find connections between concepts, a shortest path finding algorithm provides means to examine connections~\autocite{minsky1982semantic}.
Similarly, when planning a route, shortest path finding algorithms are employed. Further, when searching for some specific kind place in the surrounding like when looking for a bar, the next theater, or the next gas station, then another kind of shortest path algorithm is applied as we will see.~\autocite{bast2016route} 

But how do you make sure that these algorithms are exceptionally fast?
The bottleneck for algorithms operating on large scales of input data is mostly the time spent to load the data. 
This is due to that caches are about $50$ times faster than DDR4 RAM, which is in turn $1,000$ times faster than a solid state drive and about $100,000$ times faster then a hard disk drive~\autocite{mem-h}.
In effect, we want to minimize the number of disk IO operations that needs to be done when executing a query.
This topic has already been tackled in other types of databases, like in relational databases.
A key element to this is the concept of locality. 
The reason why caches and buffering works is the so called locality of reference~\autocite{tanenbaum2015modern, jacob2010memory}. 
That is most of the memory accesses target only a fraction of the overall data. 
Here we are going to focus on spatial locality: 
We want to order the data such that, when an element is accessed, the elements that are accessed next are within the neighbourhood of the last one. 
As disks read and write data based on chunks --- so called blocks --- packing data such that accesses remain local saves IO operations. More concretely, whenever a subsequent access stays within a block, we need one loading operation less.

In order to do this sort of reorganization, one can reorganize them statically like in relational databases. 
In relational databases this is based on the value of certain fields.
For graphs, the structure is crucial to the traversal based queries.
Thus we are going to address the issue by elaborating on static data rearrangement  methods based on the structure of the graph.

The contributions of this theis are 
\begin{itemize}
 \item a comprehensive introduction to the topic.
 \item a concise description of the problem.
 \item a pragmatic comparative metric to measure the impact of data organization on the performance of queries.
 \item a survey of existing static rearrangement methods.
 \item the proposition of two extensions to the state of the art methods:
 \begin{itemize}
  \item Use a recent community detection method to form blocks and a bandwidth reducing algorithm for ordering these blocks
  \item Reorder the incidence lists after reorganizing the data, to reestablish locality and sequential access after rearrangement.
 \end{itemize}

 \item the implementation of an in-memory graph database, traversal-based queries and the above rearrangement algorithms.
 \item an extensive evaluation of existing and the proposed method.
\end{itemize}

The rest of this thesis is organized as follows.
In the first chapter, graphs are defined formally, along with concepts based upon this definition which we are going to use throughout the thesis. 
Further data structures and algorithms are described.
Next we elaborate on the architecture of databases and a common data model for graph databases. The second chapter concludes with an example of a popular graph database called Neo4J.
After setting the context, locality is defined and an explicit problem definition is given.
Recent methods and an extension of those are discussed then, concluded by an experimental evaluation.
